\documentclass{beamer}
\usepackage[utf8]{inputenc}
\usepackage{graphicx}
\usepackage{kotex}

\usetheme{Berlin}

\title{HTTP의 기본 동작과 PHP기초}
\author{허강준 @ Team 건물주}
\institute{충남대학교 정보보호동아리 ARGOS}
\date{2021. 04. 03}

\begin{document}

\begin{frame}
    \begin{center}
        \includegraphics[height=1.5cm]{../Images/logo.png}
    \end{center}

    \maketitle
\end{frame}

\section{HTTP에 대한 이해}
    \begin{frame}{OSI 7계층}

    \end{frame}

    \begin{frame}{HTTP 패킷 구조}

    \end{frame}

    \begin{frame}{클라이언트와 서버}

    \end{frame}

    \begin{frame}{웹 서버}

    \end{frame}

    \begin{frame}{HTTPS? TLS?}

    \end{frame}

\section{HTML vs PHP}
    \begin{frame}{HTML?}

    \end{frame}

    \begin{frame}{HTML 문서의 기본 구조}

    \end{frame}

    \begin{frame}{PHP와의 관계}

    \end{frame}

    \begin{frame}{PHP의 동작 방식}

    \end{frame}

\section{개발환경 구축하기}
    \begin{frame}{Visual Studio Code 이용하기}

    \end{frame}

    \begin{frame}{문법 체크, 인텔리센스 활성화하기}

    \end{frame}

    \begin{frame}{로컬 개발 서버 활성화하기}

    \end{frame}

\section{Project}
    \begin{frame}{게시판 프로그램}

    \end{frame}

    \begin{frame}{질문?}

    \end{frame}

\end{document}