\documentclass{beamer}
\usepackage{graphicx}
\usepackage{listings}
\usepackage{kotex}

\usetheme{Berlin}

\title{Form과 Method, 헤더}
\author{허강준}
\institute{충남대학교 정보보호동아리 ARGOS}
\date{2021. 04. 10}

\begin{document}

\begin{frame}
    \begin{center}
        \includegraphics[height=1.5cm]{../Images/logo.png}
    \end{center}

    \maketitle
\end{frame}

\section{Form & HTTP Method}
    \begin{frame}{Review: HTTP 패킷의 구조}

    \end{frame}

    \begin{frame}{HTML Form 태그}

    \end{frame}

    \begin{frame}{HTML Form Input 태그}

    \end{frame}

    \begin{frame}{HTML Form Input: text}

    \end{frame}

    \begin{frame}{HTML Form Input: password}

    \end{frame}

    \begin{frame}{HTML Form Input: checkbox}

    \end{frame}

    \begin{frame}{HTML Form Input: radio}

    \end{frame}

    \begin{frame}{HTML Form Input: submit}

    \end{frame}

\section{HTTP Method}

    \begin{frame}{Method? GET? POST?}

    \end{frame}

    \begin{frame}{HTTP GET}

    \end{frame}

    \begin{frame}{HTTP GET in PHP}

    \end{frame}

    \begin{frame}{HTTP POST}

    \end{frame}

    \begin{frame}{HTTP POST in PHP}

    \end{frame}

\section{Header}

\section{마무리}
    \begin{frame}{실습: PHP Calculator}
        \begin{itemize}
            \item 폼 POST 요청을 사용하여 간단한 계산기를 구현해보세요.
            \item 파일명은 \texttt{calculator.php}와 \texttt{calculator_proc.php}를 사용할 것
            \item 덧셈, 뺄셈, 나눗셈, 곱셈이 가능해야합니다!
            \item 기한: 다음주 활동시간까지
        \end{itemize}
    \end{frame}


    \begin{frame}{질문?}

    \end{frame}

\end{document}